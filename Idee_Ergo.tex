\documentclass[12pt]{cours}

\title{\textbf{\textsc{Idée de projet mathématiques-informatique}}}
\author{DELAY -- Paul Constans}

%\corrigetrue

\parskip 0.5cm % Distance entre les paragraphes
%\parindent 0.5cm % Décalage du début de paragraphe

\begin{document}

Le point de départ est le jeu \href{https://www.catalystgamelabs.com/ergo/}{Ergo}. En gros, le but du jeu est de prouver sa propre existence. Les règles précises sont disponibles sur \url{https://www.catalystgamelabs.com/download/Ergo%20Rules%202015.pdf}. Mon idée serait de faire une implémentation de ce jeu qui pourrait permettre de jouer sur ordinateur, voir de jouer contre l'ordinateur.

Je ne sais pas exactement quel niveau de détail vous attendez pour le projet. Je n'ai pour l'instant pas trop avancé dans ma réflexion ne sachant pas si l'idée sera retenue ou non. J'ai aussi quelques doutes sur l'aspect légal : le jeu n'étant pas libre, je ne sais pas s'il est possible de l'adapter. Je pense cependant que tant que je n'ai pas l'intention de distribuer le résultat, il ne doit pas y avoir de problème ? Peut-être le Big Boss peut-il demander son avis au Directeur Technique ou au service juridique ?

Concrètement, les différents points à programmer seraient :
\begin{itemize}
\item Analyser une proposition pour vérifier si elle est syntaxiquement correcte;
\item Analyser les propositions pour déterminer quelles sont les variables \og prouvées \fg. J'imagine pour cette étape (sans avoir pour l'instant trop creusé) qu'il va falloir décrire chaque proposition sous forme d'arbre, ou peut-être passer par une forme conjonctive normale ou \dots ? ;
\item Créer une interface graphique;
\item Éventuellement implémenter une \og intelligence artificielle \fg{} pour pouvoir jouer contre l'ordinateur.
\end{itemize}

\bigskip
Si cette idée plaît au Big Boss, je peux réécrire un document plus présentable avec un cahier des charges plus précis.

\end{document}

